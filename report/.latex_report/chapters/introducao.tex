A técnica de meios-tons (halftone) consiste na criação de padrões formados por pontos pretos e brancos para reduzir a quantidade de níveis de cinza de uma imagem monocromática. Este método é amplamente empregado por veículos de comunicação impressos, como o jornal (Figura \ref{fig:1}), onde podemos representar diversas tonalidades de cinza utilizando apenas tinta preta disposta em forma de círculos de raio variado. 

%TODO: colocar uma imagem melhor
\begin{figure}[H]
    \centering
    \includegraphics[width=0.6\linewidth]{figures/journal.png}
    \caption{Aplicação de Halftone em materiais jornalísticos \cite{journal_halftone}}
    \label{fig:1}
\end{figure}

Em meios digitais, o halftone pode ser adaptado para transformar uma imagem monocromática de 256 níveis de cinza para uma com apenas pixels pretos e brancos, sendo muito utilizado para visualização de imagens e impressões. 

Além dessas aplicações clássicas, esta técnica pode ser empregada, por exemplo, na criação de mensagens criptografadas~\cite{cripto} e de marcas d'água ocultas em conteúdos autorais~\cite{watermarking}.

Neste contexto, este exercício visa aplicar algumas técnicas de meios-tons por difusão de erro muito difundidas na literatura, explorando e discutindo as principais nuances de cada abordagem. 
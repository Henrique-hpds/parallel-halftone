A técnica de meios tons (\textit{halftone}) consiste na redução de uma imagem monocromática de 256 níveis de cinza para pixels brancos e pretos, de forma que a imagem resultante fique visualmente parecida com a entrada. Um exemplo de aplicação dessa técnica pode ser visto na Figura \ref{fig:meios_tons}, em que a imagem matém a estrutura visual, porém com menos níveis de informação. Isto é útil, por exemplo, em cenários em que deseja-se economizar armazenamento, com pouco prejuízo de perca de informação visual.

\begin{figure}[H]
    \centering
    \begin{minipage}{0.45\textwidth}
        \centering
        \includegraphics[width=0.45\linewidth]{figures/intro1.png}
        \caption*{Imagem original}
    \end{minipage}
    \begin{minipage}{0.45\textwidth}
        \centering
        \includegraphics[width=0.45\linewidth]{figures/intro2.png}
        \caption*{Imagem com meios tons}
    \end{minipage}
    \caption{Comparação entre a imagem original e a imagem após aplicação da técnica de meios tons.}
    \label{fig:meios_tons}
\end{figure}

O presente trabalho tem como objetivo a paralelização de algoritmos de meios tons baseados em difusão de erro. A difusão de erro é uma técnica que distribui o erro de quantização dos pixels vizinhos, permitindo uma melhor representação visual da imagem original. A paralelização desses algoritmos pode melhorar significativamente o desempenho, especialmente em imagens de alta resolução.

Estabelece-se, portanto, como objetivo específico deste trabalho a comparação do desempenho entre as versões serial, OpenMP e CUDA de seis algoritmos consagrados na literatura: Floyd \& Steinberg, Steverson \& Arce, Burkes, Sierra, Stucki e Jarvis \& Judice \& Ninke. 
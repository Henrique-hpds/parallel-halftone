\subsection{Formulação}
\label{subsec:half:formulacao}

Dada uma imagem de entrada A com $[a_{min}, ..., a_{max}]$ níveis de cinza, desejamos aplicar um algoritmo de halftone que construa uma nova imagem B preta e branca, de forma que a imagem transformada fique visualmente parecida com a entrada. Este processo pode ser aplicado tanto para imagens em preto e branco quanto para imagens coloridas, sendo que, no caso das coloridas, o algoritmo é executado separadamente para cada canal de cor.

\textbf{\textcolor{red}{Este conteúdo está em desenvolvimento. Planejamos alterar a forma de percorrer a matriz para implementar um pipeline que permita que este código seja mais eficiente em GPUs.}}


A matriz de difusão armazena o peso que um determinado pixel possui na distribuição do erro. Para cada valor na matriz de difusão, atualizamos A de acordo com a \autoref{eq:matrizDifusao}, onde $x'$ e $y'$ representam cada coordenada da matriz de difusão. Os métodos que serão apresentados diferem apenas quanto à escolha da matriz de difusão.

\begin{equation}
A(x', \ y') \ = \ A(x', \ y') + (x', \ y') \ \cdot \ erro
\label{eq:matrizDifusao}
\end{equation}

No escopo deste projeto, serão abordados 6 métodos de difusão de erros distintos: Floyd e Steinberg; Stevenson e Arce; Burkes; Sierra; Stucki; e Jarvis, Judice e Ninke. As suas respectivas matrizes de difusão de erro estão apresentadas nas Tabelas~\ref{tab:floyd} a~\ref{tab:jjn}, onde $f(x, y)$ corresponde a $A(x, y, canal)$.

% Floyd e Steinberg
\begin{table}[H]
\centering
\renewcommand{\arraystretch}{1.5} % aumenta o espaçamento entre linhas
\setlength{\tabcolsep}{15pt} % controla o espaçamento entre colunas
\begin{tabular}{|c|c|c|}
\hline
 & \( f(x,y) \) & \( 7 \ / \ 16 \) \\ \hline
\( 3 \ / \ 16 \) & \( 5 \ / \ 16 \) & \( 1 \ / \ 16 \) \\ \hline
\end{tabular}
\caption{Floyd e Steinberg}
\label{tab:floyd}
\end{table}

% Stevenson e Arce
\begin{table}[H]
\centering
\renewcommand{\arraystretch}{1.5}
\setlength{\tabcolsep}{10pt}
\begin{tabular}{|c|c|c|c|c|c|c|}
\hline
 &&& \( f(x,y) \) && \(32 \ / \ 200\) & \\ \hline
 \(12 \ / \ 200\) && \(26 \ / \ 200\) && \(30 \ / \ 200\) && \(16 \ / \ 200\) \\ \hline
 & \(12 \ / \ 200\) && \(26 \ / \ 200\) && \(12 \ / \ 200\) & \\ \hline
 \(5 \ / \ 200\) && \(12 \ / \ 200\) && \(12 \ / \ 200\) && \(5 \ / \ 200\) \\ \hline
 
\end{tabular}
\caption{Stevenson e Arce}
\label{tab:stevenson}
\end{table}

% Burkes
\begin{table}[H]
\centering
\renewcommand{\arraystretch}{1.5}
\setlength{\tabcolsep}{12pt}
\begin{tabular}{|c|c|c|c|c|}
\hline
 && \( f(x,y) \) & \(8 \ / \ 32\) & \(4 \ / \ 32\) \\ \hline
\(2 \ / \ 32\) & \(4 \ / \ 32\) & \(8 \ / \ 32\) & \(4 \ / \ 32\) & \(2 \ / \ 32\) \\ \hline
\end{tabular}
\caption{Burkes}
\label{tab:burkes}
\end{table}

% Sierra
\begin{table}[H]
\centering
\renewcommand{\arraystretch}{1.5}
\setlength{\tabcolsep}{12pt}
\begin{tabular}{|c|c|c|c|c|}
\hline
 && \( f(x,y) \) & \(5 \ / \ 32\) & \(3 \ / \ 32\) \\ \hline
\(2 \ / \ 32\) & \(4 \ / \ 32\) & \(5 \ / \ 32\) & \(4 \ / \ 32\) & \(2 \ / \ 32\) \\ \hline
& \(2 \ / \ 32\) & \(3 \ / \ 32\) & \(2 \ / \ 32\) & \\ \hline
\end{tabular}
\caption{Sierra}
\label{tab:sierra}
\end{table}

% Stucki
\begin{table}[H]
\centering
\renewcommand{\arraystretch}{1.5}
\setlength{\tabcolsep}{12pt}
\begin{tabular}{|c|c|c|c|c|}
\hline
 && \( f(x,y) \) & \(8 \ / \ 42\) & \(4 \ / \ 42\) \\ \hline
\(2 \ / \ 42\) & \(4 \ / \ 42\) & \(8 \ / \ 42\) & \(4 \ / \ 42\) & \(2 \ / \ 42\) \\ \hline
\(1 \ / \ 42\) & \(2 \ / \ 42\) & \(4 \ / \ 42\) & \(2 \ / \ 42\) & \(1 \ / \ 42\)\\ \hline
\end{tabular}
\caption{Stucki}
\label{tab:stucki}
\end{table}

% Jarvis, Judice e Ninke
\begin{table}[h]
\centering
\renewcommand{\arraystretch}{1.5}
\setlength{\tabcolsep}{12pt}
\begin{tabular}{|c|c|c|c|c|}
\hline
 && \( f(x,y) \) & \(7 \ / \ 48\) & \(5 \ / \ 48\) \\ \hline
\(3 \ / \ 48\) & \(5 \ / \ 48\) & \(7 \ / \ 48\) & \(5 \ / \ 48\) & \(3 \ / \ 48\) \\ \hline
\(1 \ / \ 48\) & \(3 \ / \ 48\) & \(5 \ / \ 48\) & \(3 \ / \ 48\) & \(1 \ / \ 48\)\\ \hline
\end{tabular}
\caption{Jarvis, Judice e Ninke}
\label{tab:jjn}
\end{table}

\subsection{Implementação}

\subsubsection{Execução}

O código referente à implementação do halftone pode ser encontrado em \textit{1\_halftoning.py}, recebendo como parâmetro no terminal o caminho até a entrada. O programa aceita mais de uma imagem de entrada, basta passar os respectivos caminhos em sequência.  

\vspace{0.5cm}
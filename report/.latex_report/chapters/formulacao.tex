\subsection{Difusão de Erro}
\label{subsec:half:err_dif}

Dada uma imagem com 255 níveis de cinza, nosso objetivo é aplicar um algoritmo que gere uma nova imagem contendo apenas pixels pretos e brancos, de modo que o resultado final seja visualmente semelhante à imagem original. Esse procedimento pode ser realizado tanto em imagens monocromáticas quanto coloridas. No caso de imagens coloridas, o algoritmo é aplicado individualmente a cada canal de cor, convertendo o valor de cada pixel em cada canal para preto (0) ou para o valor máximo (255).  O procedimento geral pode ser encontrado no Algoritmo~\ref{alg:halftone}.

\vspace{0.5cm}

\begin{algorithm}
\caption{Algoritmo Geral de Halftone por Difusão de Erro}
\begin{algorithmic}[1] % O [1] é para numerar as linhas
\ForAll{canal $\in$ Imagem A}    
    \For{x = 0 até m - 1}
        \For{y = 0 até n - 1}
            \If{A(x, y, canal) < 128}
                \State B(x, y, canal) = 0
            \Else
                \State B(x, y, canal) = 255
            \EndIf
            \\
            \State erro = A(x, y, canal) - B(x, y, canal)
            \State DistribuirErro(A, matrizDifusao, erro)
        \EndFor
    \EndFor
\EndFor
\end{algorithmic}
\label{alg:halftone}
\end{algorithm}

As matrizes de difusão de erro são estruturas que determinam como o erro de quantização de um pixel é distribuído para seus vizinhos ainda não processados. Cada método de difusão utiliza uma matriz diferente, cujos valores representam a fração do erro a ser repassada para cada pixel vizinho. O objetivo é espalhar o erro de forma controlada, minimizando artefatos visuais e mantendo a aparência da imagem original.

Por exemplo, na matriz de Floyd e Steinberg, o erro é distribuído para quatro pixels vizinhos, com diferentes pesos, enquanto métodos como Jarvis, Judice e Ninke distribuem o erro para um número maior de vizinhos, suavizando ainda mais a transição dos tons. A escolha da matriz influencia diretamente o resultado visual do halftone e o desempenho do algoritmo.

\begin{equation}
A(x, \ y) \ = \ matrizDifusao(x, \ y) + (x, \ y) \ \cdot \ erro
\label{eq:matrizDifusao}
\end{equation}

Neste projeto, analisamos seis métodos clássicos de difusão de erro: Floyd e Steinberg, Stevenson e Arce, Burkes, Sierra, Stucki e Jarvis, Judice e Ninke. As matrizes de difusão associadas a cada método estão apresentadas nas Tabelas~\ref{tab:floyd} a~\ref{tab:jjn}, onde $f(x, y)$ representa o pixel corrente $A(x, y, canal)$. Cada matriz define como o erro de quantização é distribuído entre os vizinhos, influenciando diretamente o resultado visual do halftone.

% Floyd e Steinberg
\begin{table}[H]
\centering
\renewcommand{\arraystretch}{1.5} % aumenta o espaçamento entre linhas
\setlength{\tabcolsep}{15pt} % controla o espaçamento entre colunas
\begin{tabular}{|c|c|c|}
\hline
 & \( f(x,y) \) & \( 7 \ / \ 16 \) \\ \hline
\( 3 \ / \ 16 \) & \( 5 \ / \ 16 \) & \( 1 \ / \ 16 \) \\ \hline
\end{tabular}
\caption{Floyd e Steinberg}
\label{tab:floyd}
\end{table}

% Stevenson e Arce
\begin{table}[H]
\centering
\renewcommand{\arraystretch}{1.5}
\setlength{\tabcolsep}{10pt}
\begin{tabular}{|c|c|c|c|c|c|c|}
\hline
 &&& \( f(x,y) \) && \(32 \ / \ 200\) & \\ \hline
 \(12 \ / \ 200\) && \(26 \ / \ 200\) && \(30 \ / \ 200\) && \(16 \ / \ 200\) \\ \hline
 & \(12 \ / \ 200\) && \(26 \ / \ 200\) && \(12 \ / \ 200\) & \\ \hline
 \(5 \ / \ 200\) && \(12 \ / \ 200\) && \(12 \ / \ 200\) && \(5 \ / \ 200\) \\ \hline
 
\end{tabular}
\caption{Stevenson e Arce}
\label{tab:stevenson}
\end{table}

% Burkes
\begin{table}[H]
\centering
\renewcommand{\arraystretch}{1.5}
\setlength{\tabcolsep}{12pt}
\begin{tabular}{|c|c|c|c|c|}
\hline
 && \( f(x,y) \) & \(8 \ / \ 32\) & \(4 \ / \ 32\) \\ \hline
\(2 \ / \ 32\) & \(4 \ / \ 32\) & \(8 \ / \ 32\) & \(4 \ / \ 32\) & \(2 \ / \ 32\) \\ \hline
\end{tabular}
\caption{Burkes}
\label{tab:burkes}
\end{table}

% Sierra
\begin{table}[H]
\centering
\renewcommand{\arraystretch}{1.5}
\setlength{\tabcolsep}{12pt}
\begin{tabular}{|c|c|c|c|c|}
\hline
 && \( f(x,y) \) & \(5 \ / \ 32\) & \(3 \ / \ 32\) \\ \hline
\(2 \ / \ 32\) & \(4 \ / \ 32\) & \(5 \ / \ 32\) & \(4 \ / \ 32\) & \(2 \ / \ 32\) \\ \hline
& \(2 \ / \ 32\) & \(3 \ / \ 32\) & \(2 \ / \ 32\) & \\ \hline
\end{tabular}
\caption{Sierra}
\label{tab:sierra}
\end{table}

% Stucki
\begin{table}[H]
\centering
\renewcommand{\arraystretch}{1.5}
\setlength{\tabcolsep}{12pt}
\begin{tabular}{|c|c|c|c|c|}
\hline
 && \( f(x,y) \) & \(8 \ / \ 42\) & \(4 \ / \ 42\) \\ \hline
\(2 \ / \ 42\) & \(4 \ / \ 42\) & \(8 \ / \ 42\) & \(4 \ / \ 42\) & \(2 \ / \ 42\) \\ \hline
\(1 \ / \ 42\) & \(2 \ / \ 42\) & \(4 \ / \ 42\) & \(2 \ / \ 42\) & \(1 \ / \ 42\)\\ \hline
\end{tabular}
\caption{Stucki}
\label{tab:stucki}
\end{table}

% Jarvis, Judice e Ninke
\begin{table}[h]
\centering
\renewcommand{\arraystretch}{1.5}
\setlength{\tabcolsep}{12pt}
\begin{tabular}{|c|c|c|c|c|}
\hline
 && \( f(x,y) \) & \(7 \ / \ 48\) & \(5 \ / \ 48\) \\ \hline
\(3 \ / \ 48\) & \(5 \ / \ 48\) & \(7 \ / \ 48\) & \(5 \ / \ 48\) & \(3 \ / \ 48\) \\ \hline
\(1 \ / \ 48\) & \(3 \ / \ 48\) & \(5 \ / \ 48\) & \(3 \ / \ 48\) & \(1 \ / \ 48\)\\ \hline
\end{tabular}
\caption{Jarvis, Judice e Ninke}
\label{tab:jjn}
\end{table}

\subsection{Difusão de Erro Estocástica}

O método estocástico de difusão de erro introduz um elemento de aleatoriedade no processo tradicional de dithering. Diferente do método determinístico, que distribui o erro de quantização com base em uma matriz fixa de pesos, o método estocástico modifica a quantidade de erro difundido com base em um fator aleatório. Essa abordagem visa reduzir padrões visuais repetitivos e suavizar artefatos que podem surgir em regiões uniformes da imagem.

A implementação consiste em adicionar um ruído aleatório controlado ao valor do erro antes de sua distribuição. Formalmente, seja $e$ o erro de quantização de um pixel, e $r \sim \mathcal{U}(-p, p)$ um valor extraído de uma distribuição uniforme, com $p \in [0,1]$ sendo um parâmetro de controle da aleatoriedade. O erro ajustado $e'$ é então dado por:
\[
e' = e \cdot (1 + r)
\]
Esse erro ajustado $e'$ é então distribuído normalmente conforme a matriz de difusão escolhida (por exemplo, Floyd-Steinberg), mantendo a estrutura do algoritmo original. O valor de $p$ pode ser ajustado para controlar a intensidade da estocasticidade. Valores menores de $p$ resultam em comportamento mais próximo ao determinístico, enquanto valores maiores aumentam a variação aleatória.
\label{sec:speedup}

Nesta seção, apresentamos as métricas de Speedup obtidas durante os experimentos. O Speedup é calculado como a razão entre o tempo de execução da versão serial e o tempo de execução das versões paralelas (OpenMP e CUDA). Foram consideradas duas abordagens principais:

\begin{itemize}
    \item \textbf{Speedup por Método:} Para cada método de dithering (\textit{FloydSteinberg}, \textit{StevensonArce}, \textit{Burkes}, \textit{Sierra}, \textit{Stucki}, \textit{JarvisJudiceNinke}), calculamos o Speedup médio considerando todas as imagens processadas.
    \item \textbf{Speedup por Tamanho de Imagem:} Para cada tamanho de imagem (\textit{e.g.}, 1920x1080, 1280x720), calculamos o Speedup médio considerando todos os métodos de dithering aplicados. Os tamanhos de imagem foram classificados em quatro categorias:
        \begin{itemize}
            \item \textbf{Pequeno (small):} Imagens com largura e altura menores ou iguais a 256 pixels.
            \item \textbf{Médio (medium):} Imagens com largura e altura menores ou iguais a 512 pixels.
            \item \textbf{Grande (large):} Imagens com largura e altura menores ou iguais a 1024 pixels.
            \item \textbf{Muito Grande (xlarge):} Imagens com largura ou altura maiores que 1024 pixels.
        \end{itemize}
\end{itemize}

Os resultados obtidos estão destacados nas Tabelas~\ref{tab:speedup_methods} e~\ref{tab:speedup_sizes}. Os campos em vermelho indicam os valores que ainda precisam ser preenchidos com os dados experimentais.


\begin{table}[H]
\centering
\caption{Resultados de Speedup por Método}
\label{tab:speedup_methods}
\begin{tabular}{|c|c|c|c|c|c|}
\hline
\textbf{Método} & \textbf{Serial (ms)} & \textbf{OpenMP (ms)} & \textbf{CUDA (ms)} & \textbf{Speedup OpenMP} & \textbf{Speedup CUDA} \\ \hline
FloydSteinberg & $21.75 \pm 51.95$ & $11.92 \pm 15.31$ & $150.44 \pm 324.05$ & $1.8$ & $0.14$ \\ \hline
StevensonArce & \textcolor{red}{x.xx} & \textcolor{red}{x.xx} & \textcolor{red}{x.xx} & \textcolor{red}{x.xx} & \textcolor{red}{x.xx} \\ \hline
Burkes & \textcolor{red}{x.xx} & \textcolor{red}{x.xx} & \textcolor{red}{x.xx} & \textcolor{red}{x.xx} & \textcolor{red}{x.xx} \\ \hline
Sierra & \textcolor{red}{x.xx} & \textcolor{red}{x.xx} & \textcolor{red}{x.xx} & \textcolor{red}{x.xx} & \textcolor{red}{x.xx} \\ \hline
Stucki & \textcolor{red}{x.xx} & \textcolor{red}{x.xx} & \textcolor{red}{x.xx} & \textcolor{red}{x.xx} & \textcolor{red}{x.xx} \\ \hline
JarvisJudiceNinke & \textcolor{red}{x.xx} & \textcolor{red}{x.xx} & \textcolor{red}{x.xx} & \textcolor{red}{x.xx} & \textcolor{red}{x.xx} \\ \hline
\end{tabular}
\end{table}

\begin{table}[H]
\centering
\caption{Resultados de Speedup por Tamanho de Imagem}
\label{tab:speedup_sizes}
\begin{tabular}{|c|c|c|c|c|c|}
\hline
\textbf{Tamanho da Imagem} & \textbf{Serial (ms)} & \textbf{OpenMP (ms)} & \textbf{CUDA (ms)} & \textbf{Speedup OpenMP} & \textbf{Speedup CUDA} \\ \hline
Small & $0.75 \pm 0.13$  & $3.13 \pm 4.03$ & $9.32 \pm 1.28$ & $0.24$ & $0.08$ \\ \hline
Medium & $3.59 \pm 0.20$& $7.56 \pm 4.65$ & $33.76 \pm 2.20$ & $0.47$ & $0.11$ \\ \hline
Large & $6.52 \pm 3.56$ & $8.39 \pm 4.90$ & $60.47 \pm 27.94$ & $0.77$ & $0.11$ \\ \hline
Extra Large & $59.39 \pm 83.69$ & $21.93 \pm 24.45$ & $388.36 \pm 519.05$ & $2.71$ & $0.15$ \\ \hline
\end{tabular}
\end{table}


\begin{table}[H]
\centering
\caption{Resultados de Eficiência por Método}
\label{tab:efficiency_methods}
\begin{tabular}{|c|c|c|c|c|}
\hline
\textbf{Método} & \textbf{Eficiência OpenMP} & \textbf{Eficiência CUDA} \\ \hline
FloydSteinberg & \textcolor{red}{x.xx} & \textcolor{red}{x.xx} \\ \hline
StevensonArce & \textcolor{red}{x.xx} & \textcolor{red}{x.xx} \\ \hline
Burkes & \textcolor{red}{x.xx} & \textcolor{red}{x.xx} \\ \hline
Sierra & \textcolor{red}{x.xx} & \textcolor{red}{x.xx} \\ \hline
Stucki & \textcolor{red}{x.xx} & \textcolor{red}{x.xx} \\ \hline
JarvisJudiceNinke & \textcolor{red}{x.xx} & \textcolor{red}{x.xx} \\ \hline
\end{tabular}
\end{table}

\begin{table}[H]
\centering
\caption{Resultados de Eficiência por Tamanho de Imagem}
\label{tab:efficiency_sizes}
\begin{tabular}{|c|c|c|}
\hline
\textbf{Tamanho da Imagem} & \textbf{Eficiência OpenMP} & \textbf{Eficiência CUDA} \\ \hline
800x600 & \textcolor{red}{x.xx} & \textcolor{red}{x.xx} \\ \hline
1280x720 & \textcolor{red}{x.xx} & \textcolor{red}{x.xx} \\ \hline
1920x1080 & \textcolor{red}{x.xx} & \textcolor{red}{x.xx} \\ \hline
\end{tabular}
\end{table}

\subsection{Método estocástico}

    Os códigos de OpenMP e CUDA possuíam uma flag que permitia a utilização de um método estocástico. Esse método introduz variações aleatórias no processo de dithering, o que pode impactar os resultados de Speedup e Eficiência. Durante os experimentos, observamos que o uso do método estocástico gerou diferenças nos tempos de execução e nos resultados finais. No entanto, como o objetivo do processo de dithering é criar um efeito visual na imagem, essas variações não comprometem a qualidade visual do resultado, tornando válida a sua utilização.


    \begin{figure}[H]
        \centering
        \begin{subfigure}[b]{0.3\textwidth}
            \includegraphics[width=\textwidth]{figures/comparacao/input.png}
            \caption{Imagem Original}
            \label{fig:input_image}
        \end{subfigure}
        \hfill
        \begin{subfigure}[b]{0.3\textwidth}
            \includegraphics[width=\textwidth]{figures/comparacao/openmp.png}
            \caption{OpenMP}
            \label{fig:openmp_image}
        \end{subfigure}
        \hfill
        \begin{subfigure}[b]{0.3\textwidth}
            \includegraphics[width=\textwidth]{figures/comparacao/openmp_stochastic.png}
            \caption{OpenMP Estocástico}
            \label{fig:openmp_stochastic_image}
        \end{subfigure}
        \caption{Comparação visual entre as imagens: original, OpenMP e OpenMP com método estocástico.}
        \label{fig:comparison_images}
    \end{figure}

    \text{\color{red}Conferir se a implementacao do estocástico está correta}

    Os resultados obtidos com o método estocástico foram considerados satisfatórios. Apesar das variações introduzidas no processo de dithering, a qualidade visual das imagens geradas permaneceu consistente com os objetivos do experimento. A Figura~\ref{fig:comparison_images} ilustra que as diferenças visuais entre as imagens processadas com e sem o método estocástico são mínimas, reforçando a validade do uso dessa abordagem nos experimentos realizados.


    % Talvez seja interessante criar uma tabela com os resultados de tempos,
    % e após isso, criar uma outra tabela com os resultados do speedup,
    % considerando a média total de todos os métodos.

    \begin{table}[H]
    \centering
    \caption{Resultados de Speedup para o Método Estocástico}
    \label{tab:speedup_stochastic}
    \begin{tabular}{|c|c|c|c|c|c|}
    \hline
    \textbf{Método} & \textbf{Serial (s)} & \textbf{OpenMP (s)} & \textbf{CUDA (s)} & \textbf{Speedup OpenMP} & \textbf{Speedup CUDA} \\ \hline
    FloydSteinberg & \textcolor{red}{x.xx} & \textcolor{red}{x.xx} & \textcolor{red}{x.xx} & \textcolor{red}{x.xx} & \textcolor{red}{x.xx} \\ \hline
    StevensonArce & \textcolor{red}{x.xx} & \textcolor{red}{x.xx} & \textcolor{red}{x.xx} & \textcolor{red}{x.xx} & \textcolor{red}{x.xx} \\ \hline
    Burkes & \textcolor{red}{x.xx} & \textcolor{red}{x.xx} & \textcolor{red}{x.xx} & \textcolor{red}{x.xx} & \textcolor{red}{x.xx} \\ \hline
    Sierra & \textcolor{red}{x.xx} & \textcolor{red}{x.xx} & \textcolor{red}{x.xx} & \textcolor{red}{x.xx} & \textcolor{red}{x.xx} \\ \hline
    Stucki & \textcolor{red}{x.xx} & \textcolor{red}{x.xx} & \textcolor{red}{x.xx} & \textcolor{red}{x.xx} & \textcolor{red}{x.xx} \\ \hline
    JarvisJudiceNinke & \textcolor{red}{x.xx} & \textcolor{red}{x.xx} & \textcolor{red}{x.xx} & \textcolor{red}{x.xx} & \textcolor{red}{x.xx} \\ \hline
    \end{tabular}
    \end{table}

    \begin{table}[H]
    \centering
    \caption{Resultados de Eficiência para o Método Estocástico}
    \label{tab:efficiency_stochastic}
    \begin{tabular}{|c|c|c|}
    \hline
    \textbf{Método} & \textbf{Eficiência OpenMP} & \textbf{Eficiência CUDA} \\ \hline
    FloydSteinberg & \textcolor{red}{x.xx} & \textcolor{red}{x.xx} \\ \hline
    StevensonArce & \textcolor{red}{x.xx} & \textcolor{red}{x.xx} \\ \hline
    Burkes & \textcolor{red}{x.xx} & \textcolor{red}{x.xx} \\ \hline
    Sierra & \textcolor{red}{x.xx} & \textcolor{red}{x.xx} \\ \hline
    Stucki & \textcolor{red}{x.xx} & \textcolor{red}{x.xx} \\ \hline
    JarvisJudiceNinke & \textcolor{red}{x.xx} & \textcolor{red}{x.xx} \\ \hline
    \end{tabular}
    \end{table}